\documentclass[oneside]{book}

\usepackage[left=2cm,right=2cm,top=2cm,bottom=2cm]{geometry} 

\usepackage[utf8]{inputenc}   % otra alternativa para los caracteres acentuados y la "ñ"
\usepackage[           spanish % para poder usar el español
                      ,es-tabla % para los captions de las tablas
                       ]{babel}   
\decimalpoint %para usar el punto decimal en vez de coma para los números con decimales

%\usepackage{beton}
%\usepackage[T1]{fontenc}

\usepackage{parskip}
\usepackage{xcolor}

\usepackage{booktabs,longtable,array}
\newcolumntype{L}[1]{>{\raggedright\let\newline\\\arraybackslash\hspace{0pt}}m{#1}}
\newcolumntype{C}[1]{>{\centering\let\newline\\\arraybackslash\hspace{0pt}}m{#1}}
\newcolumntype{R}[1]{>{\raggedleft\let\newline\\\arraybackslash\hspace{0pt}}m{#1}}

\usepackage{caption}

\usepackage{enumerate} % paquete para poder personalizar fácilmente la apariencia de las listas enumerativas

\usepackage{graphicx} % figuras
\usepackage{subfigure} % subfiguras

\usepackage{amsfonts}
\usepackage{amsmath}

\usepackage{colortbl}

\usepackage{listings}
\lstset
{ %Formatting for code in appendix
    language=python,
    basicstyle=\footnotesize,
    stepnumber=1,
    showstringspaces=false,
    tabsize=1,
    breaklines=true,
    breakatwhitespace=false,
}

\definecolor{softpink}{rgb}{1,0.8,1}
	
\usepackage{float} % para controlar la situación de los entornos flotantes

\restylefloat{figure}
\restylefloat{table} 
\setlength{\parindent}{0mm}


\usepackage[bookmarks=true,
            bookmarksnumbered=false, % true means bookmarks in 
                                     % left window are numbered
            bookmarksopen=false,     % true means only level 1
                                     % are displayed.
            colorlinks=true,
            allcolors=blue,
            urlcolor=blue]{hyperref}
\definecolor{webblue}{rgb}{0, 0, 0.5}  % less intense blue

\renewcommand{\thesection}{\arabic{section}}

\title{\Huge Inteligencia de Negocio: Práctica 3 \\ Competición de Kaggle\vspace{10mm}}

\author{\huge David Cabezas Berrido \vspace{10mm} \\ 
  \huge Grupo 2: Viernes \vspace{10mm} \\ \huge dxabezas@correo.ugr.es \vspace{10mm}}

\begin{document}
\maketitle
\tableofcontents

\pagebreak

\section{Pruebas realizadas}
\setlength\LTleft{-0.5in}
\setlength\LTright{-2in}
\begin{longtable}{|c|C{2.1cm}|c|c|c|C{4cm}|C{6cm}|}
\toprule
Intento & Fecha-hora & Posición & Validación & Test & Preprocesado & Modelo \\

\midrule

1 & 19/12/2020 14:35:08 & 5 & 0.8185 & 0.7420 & Preprocesado 1 & RandomForest por defecto \\

\arrayrulecolor{black!10}\midrule

2 & 20/12/2020 10:51:19 & 4-5 & 0.8900 & 0.7580 & Preprocesado 2 & RandomForest por defecto \\

\arrayrulecolor{black!10}\midrule

3 & 20/12/2020 12:34:18 & 4-5 & 0.9040 & 0.7498 & Preprocesado 2 eliminando de train los ejemplos que inicialmente tenían descuento & RandomForest con 350 estimadores de profundidad máxima 20 \\

\arrayrulecolor{black!10}\midrule

4 & 20/12/2020 17:05:16 & 3-4 & 0.9179 & 0.7645 & Preprocesado 3 & MLPClassifier con capas ocultas de tamaño (200,200) \\

\arrayrulecolor{black!10}\midrule

5 & 21/12/2020 09:47:11 & 3-4 & 0.8750 & 0.6877 & Preprocesado 3 & 2-NN con distancia Manhattan y pesos inversamente proporcionales a la distancia \\

\arrayrulecolor{black!10}\midrule

6 & 21/12/2020 13:30:36 & 3 & 0.9143 & 0.7739 & Preprocesado 3 & C-SVM con C=65 y kernel RBF \\

\arrayrulecolor{black!10}\midrule

7 & 21/12/2020 15:32:20 & 3 & 0.9144 & 0.7645 & Preprocesado 3 & RandomForest por defecto \\

\arrayrulecolor{black!10}\midrule

8 & 22/12/2020 09:49:44 & 2 & 0.9143 & 0.8059 & Preprocesado 3 & GradientBoosting con 500 estimadores \\

\arrayrulecolor{black!10}\midrule

9 & 22/12/2020 09:50:03 & 2 & 0.9304 & 0.8016 & Preprocesado 3 & Stacking: - RandomForest por defecto - MLPClassifier con capas ocultas (200,200) - C-SVM con C=65 y kernel RBF - GradientBoosting con 500 estimadores \\

\arrayrulecolor{black!10}\midrule

10 & 22/12/2020 14:40:39 & 2 & 0.9312 & 0.7636 & Preprocesado 3 & AdaBoost con 500 árboles de profundidad 12 y learning rate de 1.1 \\

\arrayrulecolor{black!10}\midrule

11 & 23/12/2020 10:24:20 & 2 & 0.9163 & 0.7990 & Preprocesado 3 sustituyendo los valores perdidos por 0 en la columna descuento en lugar de eliminar la columna & GradientBoosting con 500 estimadores \\

\arrayrulecolor{black!10}\midrule

12 & 23/12/2020 12:12:38 & 2 &  & 0.7998 & Preprocesado 3 & Moda (predicción más frecuente) de los intentos 4, 6, 7, 8, 9, 10 y 11 \\

\arrayrulecolor{black!10}\midrule

13 & 24/12/2020 19:54:10 & 2 & 0.9205 & 0.7886 & Preprocesado 3 & GradientBoosting con 500 árboles de profundidad 6, learning rate de 1.175 y submuestras del 70\% \\

\arrayrulecolor{black!10}\midrule

14 & 25/12/2020 09:59:17 & 4 & 0.8535 & 0.7239 & Preprocesado 3 + PCA con 0.95 de varianza explicada & GradientBoosting con 500 estimadores \\

\arrayrulecolor{black!10}\midrule

15 & 25/12/2020 10:25:21 & 4 & 0.9159 & 0.8007 & Preprocesado 3 tras corregir error en LabelEncoder & GradientBoosting con 500 estimadores \\

\arrayrulecolor{black!10}\midrule

16 & 25/12/2020 11:33:44 & 4 &  & 0.8093 & Preprocesado 3 & Stacking de cuatro GradientBoosting con número de estimadores 450, 500, 550 y 600 respectivamente; tasas de aprendizaje 0.14, 0.12, 0.1, 0.08 respectivamente; todos con submuestras del 90\% \\

\arrayrulecolor{black!10}\midrule

17 & 26/12/2020 10:58:03 & 4 &  & 0.8085 & Preprocesado 3 & Stacking anterior con la opción passthrough \\

\arrayrulecolor{black!10}\midrule

18 & 26/12/2020 11:12:21 & 4 & 0.9268  & 0.7886 & Preprocesado 3 & HistGradientBoosting por defecto \\

\arrayrulecolor{black!10}\midrule

19 & 26/12/2020 11:31:13 & 4 & & 0.8024 & Preprocesado 3 & Stacking de tres
GradientBoosting con 500, 550 y 600 estimadores respectivamente; tasas
de aprendizaje 0.12, 0.1 y 0.08 respectivamente; todos con submuestras
del 90\%. También tres HistGradientBoosting con 100, 150 y 200 iteraciones máximas respectivamente \\

\arrayrulecolor{black!10}\midrule

20 & 27/12/2020 11:00:35 & 4 &  & 0.8016 & Preprocesado 3 & Stacking de cuatro GradientBoosting con número de estimadores 450, 500, 550 y 600 respectivamente; tasas de aprendizaje 0.14, 0.12, 0.1, 0.08 respectivamente; todos con submuestras del 90\%. También dos HistGradientBoosting con 100 y 200 iteraciones máximas respectivamente \\

\arrayrulecolor{black!10}\midrule

21 & 27/12/2020 13:54:7 & 4 & (0.8392) & 0.7886 & Preprocesado 3 & GradientBoosting con 550 árboles de profundidad 2, tasa de aprendizaje de 0.15 y submuestras del 90\% \\

\arrayrulecolor{black!10}\midrule

22 & 27/12/2020 14:30:41 & 4 & (0.8462) & 0.7790 & Preprocesado 3 & LightGBM con 125 árboles de profundidad máxima 8 y 27 nodos hoja como máximo; tasa de aprendizaje del 0.08 \\

\arrayrulecolor{black!10}\midrule

23 & 28/12/2020 09:21:46 & 4 & 0.9290 & 0.7808 & Preprocesado 3 & LightGBM con 200 árboles con profundiad máxima 14 \\

\arrayrulecolor{black!10}\midrule

24 & 28/12/2020 09:22:13 & 4 & 0.9291 & 0.7843 & Preprocesado 3 & LightGBM con 125 árboles con 29 nodos hoja como máximo; tasa de aprendizaje del 0.11 \\

\arrayrulecolor{black!10}\midrule

25 & 28/12/2020 09:16:14 & 4 & 0.9266 & 0.7817 & Preprocesado 3 & LightGBM por defecto \\

\arrayrulecolor{black!10}\midrule

26 & 29/12/2020 10:42:25 & 4 & \begin{tabular}[c]{c} (0.8151) \\ 0.8990  \end{tabular} & 0.7964 & Preprocesado 3 & MLP con early stopping \\

\arrayrulecolor{black!10}\midrule

27 & 29/12/2020 10:52:45 & 4 & \begin{tabular}[c]{c} (0.7920) \\ 0.9135  \end{tabular}  & 0.778 & Preprocesado 3 & SVM con C=40 \\

\arrayrulecolor{black!10}\midrule

28 & 29/12/2020 13:27:51 & 4 & \begin{tabular}[c]{c} (0.8352) \\ 0.9207 \end{tabular}  & 0.8016 & Preprocesado 3 & XGBoost con 200 árboles de profundidad 3 \\

\arrayrulecolor{black!10}\midrule

29 & 30/12/2020 09:45:53 & 4 & \begin{tabular}[c]{c} (0.8370) \\ 0.9262  \end{tabular}  & 0.7929 & Preprocesado 3 & HistGradientBoosting con 75 iteraciones máximas \\

\arrayrulecolor{black!10}\midrule

30 & 30/12/2020 09:43:11 & 4 & 0.9300  & 0.7774 & Preprocesado 3 & HistGradientBoosting con 200 iteraciones máximas, tasa de aprendizaje del 0.08 y árboles con 29 nodos hoja como máximo \\

\arrayrulecolor{black!10}\midrule

31 & 30/12/2020 10:08:35 & 4 & 0.9304  & 0.8110 & Preprocesado 3 & Stacking de GradientBoosting con 500 estimadores, MLP con early stopping, XGBoost con 200 árboles de profundidad 3 y HistGradientBoosting con 75 iteraciones máximas \\

\arrayrulecolor{black!10}\midrule

\rowcolor{softpink}

32 & 31/12/2020 10:14:38 & 5 & 0.9268  & 0.8162 & Preprocesado 3 & Stacking de GradientBoosting con 500 estimadores, MLP con early stopping y XGBoost con 200 árboles de profundidad 3 \\

\arrayrulecolor{black!10}\midrule

33 & 31/12/2020 11:29:18 & 5 & 0.9225  & 0.8067 & Preprocesado 3 & Stacking de GradientBoosting con 500 estimadores y y XGBoost con 200 árboles de profundidad 3 \\

\arrayrulecolor{black}\bottomrule
\caption{Pruebas realizadas}
\label{tab:pruebas}
\end{longtable}

\end{document}
