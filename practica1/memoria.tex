\documentclass{article}

\usepackage[left=2cm,right=2cm,top=2cm,bottom=2cm]{geometry} 

\usepackage[utf8]{inputenc}   % otra alternativa para los caracteres acentuados y la "ñ"
\usepackage[           spanish % para poder usar el español
                      ,es-tabla % para los captions de las tablas
                       ]{babel}   
\decimalpoint %para usar el punto decimal en vez de coma para los números con decimales

%\usepackage{beton}
%\usepackage[T1]{fontenc}

\usepackage{parskip}
\usepackage{xcolor}

\usepackage{caption}

\usepackage{enumerate} % paquete para poder personalizar fácilmente la apariencia de las listas enumerativas

\usepackage{graphicx} % figuras
\usepackage{subfigure} % subfiguras

\usepackage{amsfonts}
\usepackage{amsmath}

\definecolor{gris}{RGB}{220,220,220}
	
\usepackage{float} % para controlar la situación de los entornos flotantes

\restylefloat{figure}
\restylefloat{table} 
\setlength{\parindent}{0mm}


\usepackage[bookmarks=true,
            bookmarksnumbered=false, % true means bookmarks in 
                                     % left window are numbered
            bookmarksopen=false,     % true means only level 1
                                     % are displayed.
            colorlinks=true,
            allcolors=blue,
            urlcolor=blue]{hyperref}
\definecolor{webblue}{rgb}{0, 0, 0.5}  % less intense blue


\title{\Huge Inteligencia de Negocio: Práctica 1 \\ Resolución de problemas
  de clasificación y análisis experimental\vspace{10mm}}

\author{\huge David Cabezas Berrido \vspace{10mm} \\ 
  \huge Grupo 2: Viernes \vspace{10mm} \\ \huge dxabezas@correo.ugr.es \vspace{10mm}}

\begin{document}
\maketitle
\newpage
\tableofcontents
\newpage

\section{Introducción}

En esta primera práctica abordaremos el problema de decidir si un
tumor en una mamografía es benigno o maligno, por lo que deducimos que
se trata de un problema de clasificación binaria (dos clases: benigno
y maligno). Disponemos para ello de datos de 961 pacientes, para cada
uno de ellos se han medido 6 atributos entre cualitativos y
cuantitativos.

Sobre este problema real, pondremos a prueba los distintos algoritmos
estudiados en teoría y las herramientas de prácticas. Trataremos de
mejorar los resultados mediante un procesado básico de los datos y
probando distintos hiperparámetros en los algoritmos. Finalmente
compararemos los algoritmos entre sí y discutiremos cuál es el más
adecuado para el problema.

Para la experimentación usaremos validación cruzada de 5
particiones. La matriz de confusión se calculará como la suma de las
matrices de confusión de cada partición, de esta forma la matriz
resultante tendrá en cuenta cada instancia una sola vez. Como usamos
validación cruzada, nos ahorramos tener que separar un conjunto de
test para pruebas y aprovechamos cada instancia tanto para
entrenamiento como para evaluación.

Empezamos analizando el dataset. Como ya hemos comentado, consta de
961 instancias y cuenta con 6 atributos: 

\begin{enumerate}
\item Código \href{https://es.wikipedia.org/wiki/BI-RADS}{BI-RADS}: se
  trata de un número entero entre 0 y 6 (ambos incluidos) asignado por
  un radiólogo tras interpretar la mamografía. Un mayor valor
  significa una mayor probabilidad de malignidad, a excepción del
  valor 0, que indica que la información de la radiografía es
  insuficiente. Tendremos que tener esto en cuenta, ya que puede
  ``confundir'' a algunos algoritmos como el KNN, que interpretaría
  que un código 0 está más próximo a un código 1 que a un código 4, lo
  cual no tiene sentido a priori (sin atender a más características).

\item Edad del paciente: entero positivo.
\item Forma del tumor: nominal, 4 posibles formas distintas (R, O, L,
  I) y N para indicar que la forma no está definida.
\item Margen de la masa: nominal, 5 posibles valores (del 1 al 5).
\item Densidad de la masa: entero entre 1 y 4 (ambos inclusive), un
  menor valor indica mayor densidad.
\item Severidad: variable objetivo a predecir, benigno o maligno. 
\end{enumerate}

Hay algunos datos perdidos en el dataset: \vspace{-5mm}
\begin{table}[H]
  \centering
  \begin{tabular}{c|c}
    BI-RADS     & 2 \\
    Age         & 5 \\
    Shape       & 0 \\
    Margin      & 48 \\
    Density     & 76 \\
    Severity    & 0
  \end{tabular}
\end{table}
\vspace{-8mm} En principio, no hay los suficientes datos perdidos como
para dejar de considerar alguna variable, pero durante el procesado de
datos discutiremos si eliminar alguna o cómo imputar los datos
perdidos.

De los 961 instancias, 445 pertenecen a la clase maligno (46.3\%) y
516 a benigno (53.7\%). Las clases están bastante balanceadas, lo que
en genral convierte a la accuracy en una métrica bastante adecuada
para la bondad de los algoritmos de clasificación. Sin embargo, en este
problema concreto, parece mucho más grave un falso negativo
(predecimos benigno y enviamos a un paciente a casa que debería
empezar a tratarse) que un falso positivo (predecimos maligno y el
paciente irá a análisis posteriores donde probablemente se percaten
del error); por lo que puede ser interesante considerar otras métricas
de evaluación.

\section{Procesado de datos}

\section{Configuración de algoritmos}

\section{Resultados obtenidos}

\section{Análisis de resultados}

\section{Interpretación de resultados}

\section{Contenido adicional}

\section{Bibliografía}

http://faculty.marshall.usc.edu/gareth-james/ISL

\end{document}
